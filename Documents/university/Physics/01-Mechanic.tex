\documentclass[12px, a4paper, oneside]{article}
\usepackage[%
    left=1.5cm,%
    right=1.5cm,%
    top=2.0cm,%
    bottom=2.0cm,%
    % paperheight=11in,%
    % paperwidth=8.5in%
]{geometry}%
\usepackage{tikz,pgfplots}
\usepackage{mathtools}
\usepackage{hyperref}
\usepackage{xepersian}
\usetikzlibrary{3d}
\usetikzlibrary{calc}

\settextfont{Vazirmatn}

\begin{document}
\section*{مکانیک نیوتونی}
%  commands and variables:
\newcommand\deltavec[1]{\vec{\Delta{#1}}}
\newcommand\undersetrtl[2]{
${\underset{#1}{#2}}$
}
فرض کنید ذره ای به جرم
\footnote[1]{
    {\bf جرم گرانشی:}
    بر همکنش ماده و ماده,
    {\bf جرم اینرسی:}
    بر همکنش ماده و مکان, فیزیک نوین معتقد است هردو برابر هستند.
}
 m را میخواهیم در فضا مشخص کنیم.  میخواهیم حرکت آنرا بررسی کنیم. 
\subsection*{دستگاه دکارتی}
قدم اول این است که یک دستگاه مختصات مناسب انتخاب کنیم. دستگاه مختصات دکارتی دستگاهی مناسب برای این کار است.
دستگاه دکارتی دستگاهی با محور های عمود و راستگرد است. میتواند ۲ یا بیشتر محور داشته باشد.

در شکل زیر ذره m میخواهد از نقطه A به نقطه B برود. 
C را که یک منحنی جهت دار است مسیر حرکت مینامند.

\begin{tikzpicture}[
     x={(-0.5cm,-0.5cm)},y={(1cm,0cm)},z={(0cm,1cm)},        % cavalier axes 
%  x={(-0.86cm,-0.5cm)},y={(0.86cm,-0.5cm)},z={(0cm,1cm)}  % isometric axes
  ]
  % z,x,y
  \def\ha{3}  % A height
  \coordinate (O) at (0,0,0);
  \coordinate (A) at (0.5,0,2);
  \coordinate (m) at (A);
  \coordinate (B) at (0,2,2);
  \draw[-latex] (O) -- (\ha,0,0) node [left]  {$z$};
  \draw[-latex] (O) -- (0,\ha,0) node [right] {$x$};
  \draw[-latex] (O) -- (0,0,\ha) node [right] {$y$};
  \draw[blue] (A) to[out=-5,in=-80, distance=2.5cm] node [below] {$C$} (B);
  \node at (m) [below] {$m$};
  \fill (m) circle (2pt);
  \node at (A) [left] {$A (\underset{A}{x},\underset{A}{y},\underset{A}{z})$};
  \fill (A) circle (2pt);
  \node at (B) [right] {$B(\underset{B}{x},\underset{B}{y},\underset{B}{z})$};
  \fill (B) circle (2pt);
\end{tikzpicture}
\subsection*{جابجایی}
وقتی ذره m از A به B میرود کمیتی به نام از A تا B تعریف میکنیم که به آن جابجایی میگوییم.
\[ \underset{AB}{\deltavec{r}} = \vec{\underset{A}{r}} - \vec{\underset{B}{r}} \]
میتوان آنرا به صورت سه جابجایی اسکالری $
(\Delta{x},\Delta{y},\Delta{z}) $
نوشت.
\subsection*{سرعت متوسط}
حاصل تقسیم جابجایی به زمان جابجایی را سرعت میانگین یا متوسط ذره میگویند. که کمیتی برداری است.
\[ \vec{\underset{av}{v}} , \vec{\bar{v}} = \frac{\deltavec{r}}{\Delta{t}}\]
\subsection*{سرعت لحظه ای}
اگر دو نقطه ی بسیار نزدیک از مسیر را انتخاب کنیم و میانگین سرعت را حساب کنیم این میانگین برابر سرعت ذره در آن منطقه است.
\[ \vec{v} = \frac{\vec{dr}}{\vec{dt}}\]
نکته: سرعت لحظه ای در هر نقطه مماس بر مسیر حرکت است و جهت آن جهت منحنی C (مسیر حرکت) است.

\subsection*{شتاب متوسط}
فرض کنید در نقطه 1 سرعت ذره 
\undersetrtl{1}{v}
و در نقطه ی 2 سرعت ذره 
\undersetrtl{2}{v}
است. سرعت به اندازه ی
\undersetrtl{}{dv}
تغییر کرده است.
\[\vec{\Delta{v}} = \vec{\underset{2}{v}} - \vec{\underset{1}{v}}\]
\[\vec{\bar{a}} = \frac{\deltavec{v}}{\Delta{t}}\]

\end{document}